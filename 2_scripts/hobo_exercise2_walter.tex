% Options for packages loaded elsewhere
\PassOptionsToPackage{unicode}{hyperref}
\PassOptionsToPackage{hyphens}{url}
%
\documentclass[
]{article}
\usepackage{amsmath,amssymb}
\usepackage{lmodern}
\usepackage{ifxetex,ifluatex}
\ifnum 0\ifxetex 1\fi\ifluatex 1\fi=0 % if pdftex
  \usepackage[T1]{fontenc}
  \usepackage[utf8]{inputenc}
  \usepackage{textcomp} % provide euro and other symbols
\else % if luatex or xetex
  \usepackage{unicode-math}
  \defaultfontfeatures{Scale=MatchLowercase}
  \defaultfontfeatures[\rmfamily]{Ligatures=TeX,Scale=1}
\fi
% Use upquote if available, for straight quotes in verbatim environments
\IfFileExists{upquote.sty}{\usepackage{upquote}}{}
\IfFileExists{microtype.sty}{% use microtype if available
  \usepackage[]{microtype}
  \UseMicrotypeSet[protrusion]{basicmath} % disable protrusion for tt fonts
}{}
\makeatletter
\@ifundefined{KOMAClassName}{% if non-KOMA class
  \IfFileExists{parskip.sty}{%
    \usepackage{parskip}
  }{% else
    \setlength{\parindent}{0pt}
    \setlength{\parskip}{6pt plus 2pt minus 1pt}}
}{% if KOMA class
  \KOMAoptions{parskip=half}}
\makeatother
\usepackage{xcolor}
\IfFileExists{xurl.sty}{\usepackage{xurl}}{} % add URL line breaks if available
\IfFileExists{bookmark.sty}{\usepackage{bookmark}}{\usepackage{hyperref}}
\hypersetup{
  pdftitle={Exercise \#2},
  pdfauthor={Your name},
  hidelinks,
  pdfcreator={LaTeX via pandoc}}
\urlstyle{same} % disable monospaced font for URLs
\usepackage[margin=1in]{geometry}
\usepackage{color}
\usepackage{fancyvrb}
\newcommand{\VerbBar}{|}
\newcommand{\VERB}{\Verb[commandchars=\\\{\}]}
\DefineVerbatimEnvironment{Highlighting}{Verbatim}{commandchars=\\\{\}}
% Add ',fontsize=\small' for more characters per line
\usepackage{framed}
\definecolor{shadecolor}{RGB}{248,248,248}
\newenvironment{Shaded}{\begin{snugshade}}{\end{snugshade}}
\newcommand{\AlertTok}[1]{\textcolor[rgb]{0.94,0.16,0.16}{#1}}
\newcommand{\AnnotationTok}[1]{\textcolor[rgb]{0.56,0.35,0.01}{\textbf{\textit{#1}}}}
\newcommand{\AttributeTok}[1]{\textcolor[rgb]{0.77,0.63,0.00}{#1}}
\newcommand{\BaseNTok}[1]{\textcolor[rgb]{0.00,0.00,0.81}{#1}}
\newcommand{\BuiltInTok}[1]{#1}
\newcommand{\CharTok}[1]{\textcolor[rgb]{0.31,0.60,0.02}{#1}}
\newcommand{\CommentTok}[1]{\textcolor[rgb]{0.56,0.35,0.01}{\textit{#1}}}
\newcommand{\CommentVarTok}[1]{\textcolor[rgb]{0.56,0.35,0.01}{\textbf{\textit{#1}}}}
\newcommand{\ConstantTok}[1]{\textcolor[rgb]{0.00,0.00,0.00}{#1}}
\newcommand{\ControlFlowTok}[1]{\textcolor[rgb]{0.13,0.29,0.53}{\textbf{#1}}}
\newcommand{\DataTypeTok}[1]{\textcolor[rgb]{0.13,0.29,0.53}{#1}}
\newcommand{\DecValTok}[1]{\textcolor[rgb]{0.00,0.00,0.81}{#1}}
\newcommand{\DocumentationTok}[1]{\textcolor[rgb]{0.56,0.35,0.01}{\textbf{\textit{#1}}}}
\newcommand{\ErrorTok}[1]{\textcolor[rgb]{0.64,0.00,0.00}{\textbf{#1}}}
\newcommand{\ExtensionTok}[1]{#1}
\newcommand{\FloatTok}[1]{\textcolor[rgb]{0.00,0.00,0.81}{#1}}
\newcommand{\FunctionTok}[1]{\textcolor[rgb]{0.00,0.00,0.00}{#1}}
\newcommand{\ImportTok}[1]{#1}
\newcommand{\InformationTok}[1]{\textcolor[rgb]{0.56,0.35,0.01}{\textbf{\textit{#1}}}}
\newcommand{\KeywordTok}[1]{\textcolor[rgb]{0.13,0.29,0.53}{\textbf{#1}}}
\newcommand{\NormalTok}[1]{#1}
\newcommand{\OperatorTok}[1]{\textcolor[rgb]{0.81,0.36,0.00}{\textbf{#1}}}
\newcommand{\OtherTok}[1]{\textcolor[rgb]{0.56,0.35,0.01}{#1}}
\newcommand{\PreprocessorTok}[1]{\textcolor[rgb]{0.56,0.35,0.01}{\textit{#1}}}
\newcommand{\RegionMarkerTok}[1]{#1}
\newcommand{\SpecialCharTok}[1]{\textcolor[rgb]{0.00,0.00,0.00}{#1}}
\newcommand{\SpecialStringTok}[1]{\textcolor[rgb]{0.31,0.60,0.02}{#1}}
\newcommand{\StringTok}[1]{\textcolor[rgb]{0.31,0.60,0.02}{#1}}
\newcommand{\VariableTok}[1]{\textcolor[rgb]{0.00,0.00,0.00}{#1}}
\newcommand{\VerbatimStringTok}[1]{\textcolor[rgb]{0.31,0.60,0.02}{#1}}
\newcommand{\WarningTok}[1]{\textcolor[rgb]{0.56,0.35,0.01}{\textbf{\textit{#1}}}}
\usepackage{graphicx}
\makeatletter
\def\maxwidth{\ifdim\Gin@nat@width>\linewidth\linewidth\else\Gin@nat@width\fi}
\def\maxheight{\ifdim\Gin@nat@height>\textheight\textheight\else\Gin@nat@height\fi}
\makeatother
% Scale images if necessary, so that they will not overflow the page
% margins by default, and it is still possible to overwrite the defaults
% using explicit options in \includegraphics[width, height, ...]{}
\setkeys{Gin}{width=\maxwidth,height=\maxheight,keepaspectratio}
% Set default figure placement to htbp
\makeatletter
\def\fps@figure{htbp}
\makeatother
\setlength{\emergencystretch}{3em} % prevent overfull lines
\providecommand{\tightlist}{%
  \setlength{\itemsep}{0pt}\setlength{\parskip}{0pt}}
\setcounter{secnumdepth}{-\maxdimen} % remove section numbering
\ifluatex
  \usepackage{selnolig}  % disable illegal ligatures
\fi

\title{Exercise \#2}
\author{Your name}
\date{date of submission}

\begin{document}
\maketitle

\hypertarget{loaded-packages}{%
\subsubsection{Loaded packages}\label{loaded-packages}}

\begin{Shaded}
\begin{Highlighting}[]
\CommentTok{\# Load all packages here, consider to mute output of this code chunk}
\CommentTok{\# https://rmarkdown.rstudio.com/lesson{-}3.html}
\end{Highlighting}
\end{Shaded}

\hypertarget{preprocessing-r-functions}{%
\subsubsection{Preprocessing / R
functions}\label{preprocessing-r-functions}}

\begin{Shaded}
\begin{Highlighting}[]
\CommentTok{\# If data preprocessing is needed, do it here.}
\CommentTok{\# If you have specific functions, placed them here.}
\CommentTok{\# If this code chunk is not needed, delete it.}
\end{Highlighting}
\end{Shaded}

\hypertarget{quality-control-procedures-4-qcps}{%
\subsection{1. Quality control procedures (4
QCPs)}\label{quality-control-procedures-4-qcps}}

\begin{Shaded}
\begin{Highlighting}[]
\CommentTok{\# Load data from Github (then eval = TRUE)}
\NormalTok{data }\OtherTok{\textless{}{-}} \FunctionTok{read\_...}\NormalTok{(}\StringTok{\textquotesingle{}http://........\textquotesingle{}}\NormalTok{)}
\FunctionTok{head}\NormalTok{(data)}
\end{Highlighting}
\end{Shaded}

\hypertarget{measurement-range-plausible-values}{%
\subsubsection{1.1 Measurement range (Plausible
values)}\label{measurement-range-plausible-values}}

\begin{Shaded}
\begin{Highlighting}[]
\CommentTok{\# code for qcp2 here, e.g.: (set eval = TRUE to activate the code chunk)}
\NormalTok{data }\SpecialCharTok{\%\textgreater{}\%}
  \FunctionTok{mutate}\NormalTok{(}\AttributeTok{var1 =}\NormalTok{ ...) }\SpecialCharTok{\%\textgreater{}\%} 
  \FunctionTok{group\_by}\NormalTok{(var2) }\SpecialCharTok{\%\textgreater{}\%} 
  \FunctionTok{mutate}\NormalTok{(}\AttributeTok{var3 =}\NormalTok{ ...) }\SpecialCharTok{\%\textgreater{}\%} 
  \FunctionTok{summarise}\NormalTok{(}\AttributeTok{var4 =}\NormalTok{ ..., }\AttributeTok{.groups =} \StringTok{\textquotesingle{}drop\textquotesingle{}}\NormalTok{) }

\NormalTok{qcp2 }\OtherTok{\textless{}{-}}\NormalTok{ data }\SpecialCharTok{\%\textgreater{}\%} \FunctionTok{summarise}\NormalTok{(...)}
\end{Highlighting}
\end{Shaded}

\textbf{Question}: How many data points are outside the measurement
range?

\textbf{Answer}:

\hypertarget{plausible-rate-of-change}{%
\subsubsection{1.2 Plausible rate of
change}\label{plausible-rate-of-change}}

\begin{Shaded}
\begin{Highlighting}[]
\CommentTok{\# code for qcp1 here}
\end{Highlighting}
\end{Shaded}

\textbf{Question}: Describe shortly how many data points failed during
this QCP and discuss whether there is a certain daytime pattern of
failure or not?

\textbf{Answer}:

\hypertarget{minimum-variability-persistence}{%
\subsubsection{1.3 Minimum variability
(Persistence)}\label{minimum-variability-persistence}}

\begin{Shaded}
\begin{Highlighting}[]
\CommentTok{\# code for qcp3 here}
\end{Highlighting}
\end{Shaded}

\textbf{Task}: Code in this section should analyses the persistance.

\hypertarget{light-intensity}{%
\subsubsection{1.4 Light intensity}\label{light-intensity}}

\begin{Shaded}
\begin{Highlighting}[]
\CommentTok{\# code for qcp4 here}
\end{Highlighting}
\end{Shaded}

\textbf{Task}: Discuss shortly how often and when during daytime the
QCP4 flags bad data. Elaborate on some reasons for your results.

\textbf{Answer}:

\hypertarget{synthesis}{%
\subsection{2. Synthesis}\label{synthesis}}

\begin{Shaded}
\begin{Highlighting}[]
\CommentTok{\# code for synthesis here}
\end{Highlighting}
\end{Shaded}

\textbf{Task}: Present a table or graph to show how many data points
fail during the four specific QCPs. Discuss shortly the reasons for
failure and compare the different QCPs against each other.

\textbf{Answer}:

\hypertarget{results}{%
\subsection{3. Results}\label{results}}

\hypertarget{result-flagging-system-10-minutes-values}{%
\subsubsection{3.1 Result (Flagging system:
10-minutes-values)}\label{result-flagging-system-10-minutes-values}}

\begin{Shaded}
\begin{Highlighting}[]
\CommentTok{\# code for results here}
\end{Highlighting}
\end{Shaded}

\textbf{Task}: At the end of the code section above you should generate
one! tibble or data.frame named \texttt{qc\_df} with all time
information, all data points (temperature and lux) and your outcomes of
the different QCPs.

\hypertarget{result-aggregate-to-hourly-series}{%
\subsubsection{3.2 Result (Aggregate to hourly
series)}\label{result-aggregate-to-hourly-series}}

\begin{Shaded}
\begin{Highlighting}[]
\CommentTok{\# code for results here}
\end{Highlighting}
\end{Shaded}

\textbf{Task}: At the end of the code section above you should generate
one! tibble or data.frame named \texttt{hobo\_hourly} with averaged
temperature values per hour or NA values (if the hour is flagged as bad
data). See exercise description for more details.

\begin{itemize}
\item
  First column: YYYY-DD-MM HH:MM:SS
\item
  Second column: Temperature values (4 digits), NA values possible
\end{itemize}

\end{document}
